\documentclass[a4paper, titlepage]{article}

\usepackage[utf8]{inputenc}
\usepackage[bookmarks]{hyperref}

\begin{document}

\title{
	Spec for TrunkTrunk-rewrite proof-of-concepts,\\
	an attempt to get rid of Rails.
}
\maketitle

\section{Demands}
The following demands/concepts should be meet and proved where applicable.

\paragraph*{Server:}
\begin{quote} %hack to get indentation without using the external enumitem package
\begin{description}
\item [Static types:] Language should have static types.
\item [Database integration:] Use of ORM or similar relevant concept. (MYSQL support is a plus because of google compute
	engine support).
\item [Productivity:] It should be possible to 'be productive' in the
	technologies chosen, meaning that some highlevel support for http
	endpoints/routing, database mapping etc. is needed. (In particular, C
	probably doesn't fit this bill).
\item [RESTful:] We want to keep TrunkTrunk as restful as possible.
\item [OAUTH/Users:] We need/want support for consuming oauth users, as well as
	nsq-oauth services.
\item [REST client:] Beyond being a REST server, our backend will need a
	viable rest-client to interact with NSQ and logging services.
\item [Unit tests:] There must adequate support for testing, including
	data-seeding for testing relevant database concepts.
\end{description}
\end{quote}

\paragraph*{Client:}
\begin{quote} %hack to get indentation without using the external enumitem package
\begin{description}
\item [SEO friendly:] For Single-Page-Application---like technologies, such as
	AngularJS, some indication of SEO-friendliness must must be illustrated.
\item [Modularity:] Must be modular (I think we meant as in: well-separated from
	the backend?).
\item [Oauth:] Must support oauth (in order to be an in-browser nsq-client?)
\end{description}
\end{quote}
\end{document}
